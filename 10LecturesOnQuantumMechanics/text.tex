\documentclass[a4paper,10pt,uplatex]{jsarticle}


% 数式
\usepackage{amsmath,amssymb,amsthm,bm}
\usepackage{mathtools}
\mathtoolsset{showonlyrefs}
\usepackage{physics}
\usepackage{siunitx}
\usepackage[thicklines]{cancel}
\usepackage{comment}
\usepackage{graphicx}
\usepackage[dvipdfmx]{color}
\usepackage{here}
\usepackage{tikz}


\begin{document}

\title{量子力学10講 復習ノート}
\author{}
\date{\today}
\maketitle

\renewcommand{\A}{\hat{A}}
\newcommand{\U}{\hat{U}}
\newcommand{\X}{\hat{X}}
\renewcommand{\P}{\hat{P}}
\newcommand{\I}{\hat{I}}
\renewcommand{\H}{\hat{H}}
\renewcommand{\N}{\hat{N}}
\renewcommand{\inf}{\infty}


\section{第9講 運動方程式}
今までは,測定に時間がかからない(一瞬で行われる,物理量に影響がない)状況で,測定する前後の一瞬を考えてきた.しかし現実の系は当然時間変化する.これを定量的に扱うことを考える.\footnote{教科書とは異なる順序で議論を組み立てる.自主ゼミで発表した.得られる結果は変わらない.}

\subsection{時間発展演算子}
まず時間を進める演算子$\U(t)$を導入する.ここで$\U(t)$はユニタリであることを要請する:$\U(t)^\dagger \U(t) = \I$.次の性質がある.
\begin{align}
    &\hat{U}(t)\hat{U}(s) = \hat{U}(t+s) \\
    &\hat{U}(0) = \hat{I} \\
    &\hat{U}(t)^\dagger = (\hat{U}(t))^{-1} = \hat{U}(-t)
\end{align}
上から,「$t$の経過と$s$の経過を合わせると$t+s$の経過になる」,「初期状態は変化をうまない」,「逆演算は逆向きの時間経過」という意味.

\subsection{シュレディンガー描像とハイゼンベルグ描像}
量子系の時間変化の記述には,シュレディンガー描像とハイゼンベルグ描像という,二通りの流儀がある:
\begin{description}
    \item[シュレディンガー描像] 状態が変化,物理量は不変.
    \item[ハイゼンベルグ描像] 物理量が変化,状態は不変.
\end{description}
シュレディンガーとハイゼンベルグは,同時期に別々に上記の描像を発表した.これらは同値な結果を与える.

シュレディンガー描像では,状態が変化し,それは次のように表される.
\begin{equation}
    \ket{\psi(t)} = \U(t)\ket{\psi(0)}
\end{equation}
このとき物理量の期待値は
\begin{align}
    \ev*{\A}_t &= \ev{\A}{\psi(t)} \\
    &= \ev*{\A}{\U(t)\psi} \\
    &= \ev{\U(t)^\dagger\A\U(t)}{\psi} = \ev{\A(t)}{\psi}
\end{align}
と表される.一行目の式がシュレディンガー描像,三行目の式がハイゼンベルグ描像を表している.

\subsection{シュレディンガー描像}
\subsubsection{確率の保存}
まず,Bornの確率公式を使って考察してみよう.時間変化を含めたBornの確率公式は以下のようになる.
\begin{equation}
    \mathbb{P}(\ket{\chi} \xleftarrow{t} \ket{\psi}) = |\braket{\chi}{\psi(t)}|^2 = |\mel{\chi}{\U(t)}{\psi}|^2
\end{equation}
これは,初期状態$\ket{\psi}$が時刻$t$には状態$\ket{\psi(t)}$に変化して,そのときに$\ket{\chi}$に瞬時に遷移する確率を表している.ここでは$\ket{\chi}$を時間によらず一定にとったが,$\ket{\chi}$も$\ket{\psi}$とともに時間変化させるとどうなるだろう?時刻$t$においてこの確率は以下のようになる.
\begin{align}
    \mathbb{P}(\ket{\chi(t)} \leftarrow \ket{\psi(t)}) = |\braket{\chi(t)}{\psi(t)}|^2 = |\braket*{\U(t)\chi}{\U(t)\psi}|^2 = |\mel{\chi}{\U(t)^\dagger \U(t)}{\psi}|^2 = |\braket{\chi}{\psi}|^2
\end{align}
つまり,遷移先の状態もともに時間変化させれば,確率は保存される.

\subsubsection{シュレディンガー方程式}
量子系の状態変化を記述する運動方程式,シュレディンガー方程式を要請する.
\begin{equation}
    i\hbar\pdv{t}\ket{\psi(t)} = \H\ket{\psi(t)}
\end{equation}
ここで$\H$はハミルトニアン(Hamiltonian)で,エネルギーの次元を持つ.ここでは,ハミルトニアンが時間に依存しない状況を考える.シュレディンガー方程式に$\ket{\psi(t)} = \U(t)\ket{\psi}$を代入する.
\begin{equation}
    i\hbar\pdv{\U(t)}{t}\ket{\psi} = \H\U(t)\ket{\psi}
\end{equation}
ここで初期状態$\ket{\psi}$は好きに決められるので,
\begin{equation}
    i\hbar\pdv{\U(t)}{t} = \H\U(t)
\end{equation}
が必要.これは$\U(t)$についての簡単な微分方程式で,初期条件$\U(0)=\I$と合わせてこれを解くと
\begin{equation}
    \U(t) = e^{-\frac{i}{\hbar}\H t}
\end{equation}
となる.ここで
\begin{equation}
    e^{-\frac{i}{\hbar}\H t} = \sum_{n=0}^\inf \frac{1}{n!} \left(-\frac{i}{\hbar}\H t\right)^n
\end{equation}
である.時間発展演算子には,ユニタリーであることを要請していた.これを確認すると,
\begin{align}
    \U(t)^\dagger \U(t) &= e^{\frac{i}{\hbar}\H^\dagger t}e^{-\frac{i}{\hbar}\H t}\\
    &= e^{\frac{i}{\hbar}(\H^\dagger - \H)t}
\end{align}
これが恒等演算子$\I$に等しいので,$\H^\dagger = \H$となる.つまり,ハミルトニアンは自己共役である.

\subsubsection{エネルギー固有状態}
ハミルトニアンは自己共役でであるから,物理量を表しうる.実際,ハミルトニアンはエネルギーを表す物理量であり,その固有値が系のエネルギーとみなされる.ハミルトニアンを実際に構成して,具体的に固有値問題を解く(エネルギーを求める)ことは次章にまわして,固有状態のときについて考えよう.固有状態ということは,すなわち状態ベクトルが
\begin{equation}
    \H \ket{\phi} = E\ket{\phi}
\end{equation}
を満たすということ.離散スペクトルならば,上式を満たすような固有ベクトルを集めてCONSが作れるので,このときは測定値は100\%で$E$になる.この固有状態の時間発展を考える.
\begin{align}
    \ket{\phi(t)} &= \U(t) \ket{\phi} \\
    &= e^{-\frac{i}{\hbar}\H t} \ket{\phi} \\
    &= e^{-\frac{i}{\hbar}Et} \ket{\phi}
\end{align}
となる.ここで三番目の等号は$e^z$の定義を振り返ればわかる.これを利用して,他の(固定した)状態への遷移確率は,
\begin{equation}
    \mathbb{P}(\ket{\chi} \xleftarrow{t} \ket{\phi}) = |\braket{\chi}{\phi(t)}|^2 = |e^{-\frac{i}{\hbar}Et}\braket{\chi}{\phi}|^2 = |\braket{\chi}{\phi}|^2
\end{equation}
となる.つまり,任意の状態への遷移確率は変化しない.この著しい特徴から,エネルギー固有状態は定常状態と呼ばれる.

\begin{comment}
    描像:時間発展演算子がユニタリー,ハミルトニアンは時間不変,シュレディンガー方程式
\end{comment}

\subsection{ハイゼンベルグ描像}
今までシュレディンガー描像で考えてきたが,ハイゼンベルグ描像を見てみよう.ハイゼンベルグ描像は,状態が不変で,物理量が変化するとみなすのだった.この物理量の変化というのは次のように表される.
\begin{equation}
    \A(t) = \U(t)^\dagger \A \U(t)
\end{equation}
ここで$\U(t)$は時間発展演算子で,シュレディンガー描像でみてきたものと性質は変わらない.シュレディンガー描像で得られた結果$\U(t) = e^{-\frac{i}{\hbar}\H t}$を用いて,物理量の時間変化を求める.
\begin{align}
    \dv{\A(t)}{t} &= \dv{\U(t)^\dagger}{t}\A\U(t) + \U(t)^\dagger \A\dv{\U(t)}{t} \\
    &= \frac{i}{\hbar} \H e^{\frac{i}{\hbar}\H t} \A e^{-\frac{i}{\hbar}\H t} - e^{\frac{i}{\hbar}\H t}\A e^{-\frac{i}{\hbar}\H t} \frac{i}{\hbar} \H \\
    &= -\frac{i}{\hbar}\left(\A(t)\H - \H\A(t)\right) = -\frac{i}{\hbar}[\A(t), \H]
\end{align}
これがハイゼンベルグ描像における基礎方程式で,ハイゼンベルグ方程式である.ここではシュレディンガー方程式からハイゼンベルグ方程式が導出された.

一方で,ハイゼンベルグ方程式を要請としてシュレディンガー方程式を導出することもできる.まず,ハイゼンベルグ方程式から,
\begin{align}
   \dv{\A(t)}{t} &= -\frac{i}{\hbar}(\U^\dagger\A\U\U^\dagger\H\U - \U^\dagger\H\U\U^\dagger\A\U) \\
   &= -\frac{i}{\hbar}\U(t)^\dagger(\A\H-\H\A)\U(t) \\
   &= \U(t)^\dagger \left[\A, -\frac{i}{\hbar}\H\right] \U(t)
\end{align}
ここで,ハミルトニアンが時間変化しないことを用いた.一方で$\A(t) = \U(t)^\dagger \A \U(t)$より
\begin{equation}
    \dv{\A(t)}{t} = \dv{\U^\dagger}{t}\A\U + \U^\dagger\A\dv{\U}{t}
\end{equation}
である.$\U(t)\U(t)^\dagger = \I$から,この両辺を時間微分して,
\begin{align}
    \dv{\U}{t}\U^\dagger + \U\dv{\U^\dagger}{t} &= 0 \\
    \U\dv{\U^\dagger}{t} &= -\dv{\U}{t}\U^\dagger
\end{align}
となることを使うと,
\begin{align}
    \dv{\A(t)}{t} &= \U^\dagger\U\dv{\U^\dagger}{t}\A\U + \U^\dagger\A\dv{\U}{t}\U^\dagger\U \\
    &= \U(t)^\dagger\left(\A\dv{\U}{t}\U^\dagger - \dv{\U}{t}\U^\dagger\A\right)\U(t) \\
    &= \U(t)^\dagger\left[\A,\dv{\U}{t}\U^\dagger\right] \U(t)
\end{align}
だから,二通りの表し方を比較して,
\begin{equation}
    \U(t)^\dagger \left[\A, -\frac{i}{\hbar}\H\right] \U(t) = \U(t)^\dagger\left[\A,\dv{\U}{t}\U^\dagger\right] \U(t)
\end{equation}
これが任意の$\A$で成り立つのだから,
\begin{align}
    -\frac{i}{\hbar}\H &= \dv{\U(t)}{t}\U(t)^\dagger \\
    -\frac{i}{\hbar}\H\U(t) &= \dv{\U(t)}{t}
\end{align}
である.この微分方程式を解けば,$U(t) = e^{-\frac{i}{\hbar}\H t}$がわかる.シュレディンガー描像では状態が時間変化する:$\ket{\psi(t)} = \U(t)\ket{\psi}$.これに先程求めた$\U(t)$を用いて時間微分すれば,シュレディンガー方程式が出てくる.

\begin{comment}
    要請:時間発展演算子がユニタリー,ハミルトニアンは不変,ハイゼンベルグ方程式
\end{comment}

\section{10講 調和振動子}
古典力学からのアナロジーで,量子力学でも調和振動子を考える.古典力学の調和振動子の力学的エネルギーは,位置$x$と運動量$p$,角振動数$\omega$によって
\begin{equation}
    E = \frac{p^2}{2m} + \frac{1}{2} m\omega^2x^2
\end{equation}
と表される.古典力学からのアナロジーで,これらをすべて演算子に置き換えたものをハミルトニアンとする.
\begin{equation}
    \H = \frac{\P^2}{2m} + \frac{1}{2} m\omega^2\X^2
\end{equation}
ハイゼンベルグ方程式を考えると,運動方程式は次のようになる.
\begin{align}
    \dv{\P}{t} &= -\frac{i}{\hbar}[\P(t),\H] = -m\omega^2\X \\
    \dv{\X}{t} &= -\frac{i}{\hbar}[\X(t),\H] = \frac{\P}{m}
\end{align}
この計算には正準交換関係$[\X, \P] = i\hbar\I$を使った.この運動方程式を解くことを考える.第一式に$i$をかけ,第二式に$m\omega$をかけたものを足し合わせると,
\begin{equation}
    \dv{t}(m\omega\X + i\P) = -i\omega(m\omega\X + i\P)
\end{equation}
となる.したがって
\begin{equation}
    m\omega\X(t) + i\P(t) = e^{-i\omega t}(m\omega\X(0) + i\P(0))
\end{equation}
両辺を比較することで解が得られる.
\begin{align}
    \X(t) &= \X(0)\cos{\omega t} + \frac{1}{m\omega}\P(0)\sin{\omega t} \\
    \P(t) &= \P(0)\cos{\omega t} - m\omega\X(0)\sin{\omega t}
\end{align}
次にハミルトニアンの固有値問題を考える.ここで先程計算で利用した式を$\A = m\omega\X + i\P$とおく.すると,
\begin{equation}
    [\A, \A^\dagger] = 2m\hbar\omega\I 
\end{equation}
となるので,これが$\I$となるように係数を付け足したものを新しく$\A$とおく:$\A = \frac{1}{\sqrt{2m\hbar\omega}}(m\omega\X+i\P)$.$[\A, \A^\dagger] = \I$である.これを用いて$\H$を書き直す.
\begin{align}
    \H &= \hbar\omega\left(\A^\dagger\A + \frac{1}{2}\I\right) \\
    &= \hbar\omega\left(\N + \frac{1}{2}\I\right)
\end{align}
ここで$\N = \A^\dagger\A$とおいた.ハミルトニアンの固有値問題は$\N$の固有値問題を解くことに置き換わった.$\N$と$\A$の性質は次のようになる.どれも簡単な計算をすればわかる.
\begin{align}
    [\N, \A] &= -\A \\
    [\N, \A^\dagger] &= \A^\dagger
\end{align}

\end{document}